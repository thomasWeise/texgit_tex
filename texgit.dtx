% \iffalse meta-comment
%
% Copyright (C) 2023--2025 by Thomas Weise
%
% This file may be distributed and/or modified under the conditions of the
% LaTeX Project Public License, either version 1.3 of this license or
% (at your option) any later version. The latest version of this license is in
% http://www.latex-project.org/lppl.txt.
% The license can be found at
% https://thomasweise.github.io/texgit_tex/LICENSE.html.
% This work is author-maintained, i.e., it can only be maintained by
% Thomas Weise.
%
% \fi
%
% \iffalse
%
%<*driver>
\ProvidesFile{texgit.dtx}
%</driver>
%<package>\NeedsTeXFormat{LaTeX2e}[1999/12/01]%
%<package>\ProvidesPackage{texgit}[2025/06/26 0.9.7 now offering file names (via gitName) and escaped file names (via gitNameEsc)]%
%
%<*driver>
\documentclass{ltxdoc}
\usepackage{texgit}[2025/06/26]%
\makeatletter%
\let\@texgit@needsTexgitPass\relax%
\makeatother%
%
\usepackage{xcolor}%
\usepackage[%
colorlinks,%
breaklinks=true,%
urlcolor={blue!80!black},%
menucolor=black,%
linkcolor={red!50!black},%
bookmarks=true,%
bookmarksopen=false,%
hyperfootnotes=true,%
citecolor={blue!50!black},%
filecolor=black,%
pdfkeywords={LaTeX, package, texgit}
]{hyperref}%
\usepackage{breakurl}%
\usepackage[square,numbers,comma,sort&compress]{natbib}%
%
\RequirePackage[utf8]{inputenc}%
\RequirePackage[T1]{fontenc}%
\RequirePackage{CJKutf8}%
\AtBeginDocument{\CJK{UTF8}{gkai}}%
%
\newcommand{\exampleautorefname}{Example}%
\renewcommand{\sectionautorefname}{Section}%
\renewcommand{\subsectionautorefname}{Section}%
\renewcommand{\subsubsectionautorefname}{Section}%
\usepackage{verbatim}%
\usepackage{graphicx}%
\usepackage{subcaption}%
\usepackage{afterpage}%
%
\usepackage{listings}%
\lstset{%
language=[LaTeX]TeX,%
basicstyle=\small,%
keywordstyle=\ttfamily\color{teal!90!black}\bfseries,%
morekeywords={url,verbatiminput,lstset,lstinputlisting},
classoffset=1,morekeywords={gitLoad,gitFile,gitArg,gitExec,gitName,gitNameEsc,gitUrl},keywordstyle=\ttfamily\bfseries\color{red!70!black},classoffset=0,%
identifierstyle=,%
commentstyle=\color{gray}\footnotesize,%
stringstyle=\ttfamily\color{red!90!black}, % typewriter type for string
numbers=left,%
numberstyle=\tiny,%
columns=fullflexible,%
frame=shadowbox,%
frameround=tttt,%
backgroundcolor=\color{black!10!yellow!5!white},%
literate=%
{á}{{\'a}}1 {é}{{\'e}}1 {í}{{\'i}}1 {ó}{{\'o}}1 {ú}{{\'u}}1%
{Á}{{\'A}}1 {É}{{\'E}}1 {Í}{{\'I}}1 {Ó}{{\'O}}1 {Ú}{{\'U}}1%
{à}{{\`a}}1 {è}{{\'e}}1 {ì}{{\`i}}1 {ò}{{\`o}}1 {ù}{{\`u}}1%
{À}{{\`A}}1 {È}{{\'E}}1 {Ì}{{\`I}}1 {Ò}{{\`O}}1 {Ù}{{\`U}}1%
{ä}{{\"a}}1 {ë}{{\"e}}1 {ï}{{\"i}}1 {ö}{{\"o}}1 {ü}{{\"u}}1%
{Ä}{{\"A}}1 {Ë}{{\"E}}1 {Ï}{{\"I}}1 {Ö}{{\"O}}1 {Ü}{{\"U}}1%
{â}{{\^a}}1 {ê}{{\^e}}1 {î}{{\^i}}1 {ô}{{\^o}}1 {û}{{\^u}}1%
{Â}{{\^A}}1 {Ê}{{\^E}}1 {Î}{{\^I}}1 {Ô}{{\^O}}1 {Û}{{\^U}}1%
{œ}{{\oe}}1 {Œ}{{\OE}}1 {æ}{{\ae}}1 {Æ}{{\AE}}1 {ß}{{\ss}}1%
{ç}{{\c c}}1 {Ç}{{\c C}}1 {ø}{{\o}}1 {å}{{\r a}}1 {Å}{{\r A}}1%
{€}{{\EUR}}1 {£}{{\pounds}}1 {«}{{\guillemotleft}}1%
{»}{{\guillemotright}}1 {ñ}{{\~n}}1 {Ñ}{{\~N}}1 {¿}{{?`}}1%
{…}{{\ldots}}1 {≥}{{>=}}1 {≤}{{<=}}1 {„}{{\glqq}}1 {“}{{\grqq}}1%
{”}{{''}}1%
{α}{{$\alpha$}}1 {β}{{$\beta$}}1 {γ}{{$\gamma$}}1 {δ}{{$\delta$}}1%
{ε}{{$\epsilon$}}1 {ζ}{{$\zeta$}}1 {η}{{$\eta$}}1 {θ}{{$\theta$}}1%
{ι}{{$\iota$}}1 {κ}{{$\kappa$}}1 {λ}{{$\lambda$}}1 {μ}{{$\mu$}}1%
{ν}{{$\nu$}}1 {ξ}{{$\xi$}}1 {ο}{{o}}1 {π}{{$\pi$}}1 {ρ}{{$\rho$}}1%
{σ}{{$\sigma$}}1 {τ}{{$\tau$}}1 {υ}{{$\upsilon$}}1 {φ}{{$\phi$}}1%
{χ}{{$\chi$}}1 {ψ}{{$\psi$}}1 {ω}{{$\omega$}}1 {Α}{{A}}1 {Β}{{B}}1%
{Γ}{{$\Gamma$}}1 {Δ}{{$\Delta$}}1 {Ε}{{E}}1 {Ζ}{{Z}}1 {Η}{{H}}1%
{Θ}{{$\Theta$}}1 {Ι}{{I}}1 {Κ}{{K}}1 {Λ}{{$\Lambda$}}1 {Μ}{{M}}1%
{Ν}{{N}}1 {Ξ}{{$\Xi$}}1 {Ο}{{O}}1 {Π}{{$\Pi$}}1 {Ρ}{{P}}1%
{Σ}{{$\Sigma$}}1 {Τ}{{T}}1 {Υ}{{Y}}1 {Φ}{{$\Phi$}}1 {Χ}{{X}}1%
{Ψ}{{$\Psi$}}1 {Ω}{{$\Omega$}}1%
{∖}{{$\setminus$}}1 {∪}{{$\cup$}}1 {∩}{{$\cap$}}1%
{≈}{{$\approx$}}1 {∈}{{$\in$}}1 {∉}{{$\notin$}}1%
}%
%
%
\gdef\tindex#1{\expandafter\index{#1}}%
\gdef\tbindex#1{\expandafter\index{\textbackslash#1}}%
%
\usepackage{placeins}%
%
\EnableCrossrefs
\CodelineIndex
\RecordChanges
%
%
\begin{document}
  \DocInput{texgit.dtx}
  \PrintChanges
  \PrintIndex
\end{document}
%</driver>
% \fi
%
% \CharacterTable
%  {Upper-case    \A\B\C\D\E\F\G\H\I\J\K\L\M\N\O\P\Q\R\S\T\U\V\W\X\Y\Z
%   Lower-case    \a\b\c\d\e\f\g\h\i\j\k\l\m\n\o\p\q\r\s\t\u\v\w\x\y\z
%   Digits        \0\1\2\3\4\5\6\7\8\9
%   Exclamation   \!     Double quote  \"     Hash (number) \#
%   Dollar        \$     Percent       \%     Ampersand     \&
%   Acute accent  \'     Left paren    \(     Right paren   \)
%   Asterisk      \*     Plus          \+     Comma         \,
%   Minus         \-     Point         \.     Solidus       \/
%   Colon         \:     Semicolon     \;     Less than     \<
%   Equals        \=     Greater than  \>     Question mark \?
%   Commercial at \@     Left bracket  \[     Backslash     \\
%   Right bracket \]     Circumflex    \^     Underscore    \_
%   Grave accent  \`     Left brace    \{     Vertical bar  \|
%   Right brace   \}     Tilde         \~}
%
% \changes{0.8.0}{2023/12/04}{the initial draft version}
% \changes{0.8.1}{2023/12/05}{slightly improved documentation}
% \changes{0.8.2}{2023/12/05}{improved texgit.tds.zip}
% \changes{0.8.3}{2024/08/01}{supporting arbitrary commands via the new |texgit_py| version}
% \changes{0.8.4}{2024/08/02}{improved build process}
% \changes{0.8.5}{2024/08/07}{improved examples: added an example with multiple git command results}
% \changes{0.8.6}{2024/09/20}{the use case of virtual environments in conjunction with the |texgit| Python package of version 0.8.17 or greater is documented}
% \changes{0.9.0}{2025/06/12}{complete overhaul of the API}
% \changes{0.9.1}{2025/06/14}{improved documentation}
% \changes{0.9.2}{2025/06/15}{gitExec with empty ID permitted}
% \changes{0.9.3}{2025/06/17}{better documentation}%
% \changes{0.9.4}{2025/06/18}{name change to texgit}%
% \changes{0.9.5}{2025/06/19}{improved documentation and slight updates}%
% \changes{0.9.6}{2025/06/20}{now forcing pdflatex to crash if Python postprocessor is not applied}%
% \changes{0.9.7}{2025/06/26}{now offering file names (via gitName) and escaped file names (via gitNameEsc)}%
%
% \GetFileInfo{texgit.dtx}
%
% \DoNotIndex{\def,\if,\fi}
%
% \title{The \textsf{texgit} Package}
% \author{Thomas Weise~(汤卫思教授)\\%
% \resizebox{0.95\textwidth}{!}{%
% \parbox{1.05\linewidth}{%
% \centering%
% Institute of Applied Optimization~(应用优化研究所, IAO)\\%
% School of Artificial Intelligence and Big Data~(人工智能与大数据学院)\\%
% Hefei University~(合肥大学)\\%
% Hefei 230601, Anhui, China~(中国安徽省合肥市)\\%
% \href{mailto:tweise@hfuu.edu.cn}{tweise@hfuu.edu.cn} $\cdot$ \href{mailto:tweise@ustc.edu.cn}{tweise@ustc.edu.cn}%
% }}}%
% \date{\today}
%
% \renewcommand{\textfraction}{0.05}%
% \renewcommand{\topfraction}{0.85}%
% \renewcommand{\bottomfraction}{0.95}%
% \renewcommand{\floatpagefraction}{0.95}%
%
% \maketitle
%
% \begin{abstract}
% This package allows you to download and access files that reside in a
% |git| repository from within your \LaTeX\ code.
% This can be used, for example, to include program code from an actual
% software in life repository in your \LaTeX\ documents.
% It allows you to postprocess these files, e.g., to apply programs that
% remove comments or reformat code and then to include these postprocessed
% files.
% It furthermore allows you to execute programs (or scripts from |git|
% repositories) on your machine and include their output into your \LaTeX\
% documents.
% Finally, it also allows you to allocate files and pass them as parameters
% to the programs that you execute.
% With this, you could create PDF figures on the fly and then include them
% into your \LaTeX\ documents.
%
% This \LaTeX\ package works \emph{only} in combination with the Python
% package |texgit|.
% To implement its functionality, it offers the following commands:%
%
% \begin{itemize}\sloppy
%
% \item |\gitLoad{id}{repoURL}{pathInRepo}{postproc}|\tbindex{gitLoad}
% loads a file |pathInRepo| from the |git|\tindex{git} repository |repoURL|,
% \emph{optionally} post-processes it by piping its contents into the standard
% input of a command |postproc| capturing its standard output.
%
% \item |\gitFile{id}|\tbindex{gitFile} provides a local path to a file
% created this way.
% Using the |\gitFile{id}| macro, you can then include the file in \LaTeX\
% directly or load it as source code listing.
%
% \item |\gitName{id}|\tbindex{gitName} provides the name of the file
% created this way.
%
% \item |\gitNameEsc{id}|\tbindex{gitName} provides the name of the file
% created this way, but with the characters |_|, |$|, and | | escaped to
% |\_|, |\$|, and |~|, respectively.
%
% \item |\gitUrl{id}|\tbindex{gitUrl} provides the URL to the original file
% in the |git| repository.
%
% \item |\gitExec{id}{repoURL}{pathInRepo}{command}|\tbindex{gitExec}
% executes an arbitrary command |command|, either in the current directory or
% inside a directory |pathInRepo| of the |git|\tindex{git} repository
% |repoURL| and fetches the standard output into a local file, the path to which
% is made available to the file again as macro
% |\gitFile{id}|\tbindex{gitFile}.
% You can also leave the |id| parameter empty.
% This makes sense if you are not interested in the output of the program.
% Maybe you just want to execute a script, say, something that creates a figure
% stored in some file.
% Without |id| parameter, you cannot access the stdout of this command.
% And sometimes, this is ok.
%
% \item |\gitArg{id}{prefix}{suffix}|\tbindex{gitArg} allocates an additional
% file, whose name will be composed of the optional |prefix| and |suffix|.
% Such files can be passed as arguments to |\gitExec| or
% |\gitLoad|\tbindex{gitLoad} by including |(?id?)| in their commands'
% argument list.
% This way, we can, for example, instruct a program to create a graphic
% and store it in a certain file that we can later load from
% |\gitFile{id}|\tbindex{gitFile}.
%
% \item |\gitIf{id}{ifDone}{ifNotDone}| executes the code |ifDone| starting
% in the second |pdflatex| pass, i.e., after the Python |texgit| package has
% been applied to the |aux| file generated during the first |pdflatex| pass.
% During the first |pdflatex| pass and before the Python |texgit| package
% was applied, |ifNotDone| will be executed.
% \end{itemize}
%
% The functionality of the package is implemented by storing the |git|
% requests in the |aux|\tindex{aux} file of the project during the first
% |pdflatex|\tindex{pdflatex} pass.
% The |aux| file is then processed by the Python package which performs the
% actual |git| queries, program executions, stores the result in local files,
% and adds the resolved paths to the |aux| file.
% Thus, during the first |pdflatex|\tindex{pdflatex} run, |\gitFile| and
% |\gitUrl| offer dummy results.
% During the second and later pass, after the Python program |texgit| has
% been applied to the |aux| file, they then provide the actual paths and
% URLs.
% In the essense, |texgit| works somewhat like \BibTeX.
% \end{abstract}
%
% \setcounter{tocdepth}{2}
% \tableofcontents
%
% \section{Introduction}%
%
% \subsection{Addressed Problem and Use Case}%
% Let's say you want to make teaching material in the field of computer
% science.
% In a wide variety of computer science fields, you may want to include source
% code examples in your lecture script or slides.
% The standard way is to either write some pseudo-code or program-like
% snippets.
% Usually these neither compile nor are they maintained well and they are,
% hence, often riddled with mistakes.
% That is not nice.
%
% What we want is to have snippets of ``real'' code.
% Code that we can compile, unit test, and run.
% Now such code naturally would not be sprinkled into our \LaTeX\ teaching
% material sources.
% That would be a mess to organize and maintain.
%
% A natural location for source code examples is a separate |git| repository.
% Maybe on \href{https://github.com}{GitHub}, maybe somewhere else.
% If I wanted to do a lecture on, say, optimization algorithms, I would like to
% have the optimization algorithms
% \href{https://thomasweise.github.io/moptipy}{implemented} in an useful
% fashion.
% I would put them into a repository where I can build and test these real
% codes as a complete and separate piece of work.
%
% Then I want to use them in my lecture scripts (written in \LaTeX) as well.
% However, I do not want to \emph{copy} them there.
% I want that my lecture scripts directly reference the |git| repository with
% the real code.
% I want them to ``include'' the examples from there.
% If I change the code in the |git| repository and then re-compile my
% teaching material, these changes should automatically be reflected there.
%
% That is the use case we tackle here.
% We offer a solution to the question
%
% \begin{quote}
% \emph{``How can we include snippets of code from a separate, complex code
% basis (located in a |git| repository) into our \LaTeX\ documents?''}
% \end{quote}
%
% Additionally, sometimes we want to execute the code from that repository and
% capture the standard output.
% This output could then be displayed as listing next to the code.
% This package also provides this functionality.
%
% Maybe we either have local programs or programs in a |git| repository that
% create complex figures or graphics.
% Our package also offers us the ability execute such programs, tell them
% where to store their output, and then allowing us to include this output
% into our \LaTeX\ documents, say, via |\includegraphics|.
%
% Our package furthermore \emph{caches} such outputs.
% If you refer to multiple files from one repository, this repository will be
% cloned only once.
% If you need to perform several \LaTeX\ passes, say, because you have a
% bibliography and a glossary and whatnot, then you need to run |texgit|
% after \emph{each} pass.
% However, |texgit| will only clone the repositories and execute the
% programs once, during its first pass.
%
% All results are cached, usually in a directory called |__git__| in your
% document's direcotry.
% Unless you delete this directory, the cache will persist, even if you
% change your \LaTeX\ document and perform an entirely new compilation.
%
% \subsection{Provided Functionality}%
% \label{sec:functionality}%
% Our package offers a combination of
% \begin{itemize}
% \item a \LaTeX\ package -- this package here -- with its sources at
% \url{https://github.com/thomasWeise/texgit_tex}, and
% \item a Python\index{Python} program, published at
% \url{https://pypi.org/projects/texgit} with sources available at
% \url{https://github.com/thomasWeise/texgit_py}.
% \end{itemize}
%
% This \LaTeX\ package provides the command |\gitLoad| that can load a
% specific file from a specific |git| repository and, optionally, pipe the
% file contents through a program for post-processing.
%
% It also provides the command |\gitExec|, which can execute either a local
% program or a program loaded from a git repository and fetch its output.
%
% The additional command |\gitArg| allows you to create files whose paths
% are passed as arguments to programs, which may be useful to create graphics
% and other non-textual output.
%
% All three types of requests are stored in the |aux| file during the first
% |pdflatex| pass, then resolved by the Python program, and their results
% become available in the second |pdflatex| pass via the commands |\gitFile|
% and |\gitUrl|.
%
% \section{Usage}%
% Using the package requires the following steps:
% \begin{enumerate}%
%%
% \item Obviously, both the
% \href{https://thomasweise.github.io/texgit_tex}{\LaTeX\ package}
% \emph{and} its
% \href{https://thomasweise.github.io/texgit_py}{Python companion}
% package must be installed (see \autoref{sec:installation}).
%%
% \item In your document, you need to load the package in the preamble
% (see \autoref{sec:loadThePackage}).%
%%
% \item Then you can make |git| queries, e.g., via |\gitLoad{id}...|
% or |\gitExec{id}...| (see \autoref{sec:gitquery}).
%
% \item At this stage, |\gitFile{id}| and |\gitUrl{id}| will hold dummy
% results, meaning that you can still use these commands but they will
% not yet provide useful data.
%%
% \item The Python post-processor package can carry them out after the
% first |pdflatex| run.
%
% \item In the next |pdflatex| run, |\gitFile{id}| and |\gitUrl{id}|
% are defined appropriately, see \autoref{sec:pythonProgram}.
% \end{enumerate}
%
% If your \LaTeX\ document is called |article.tex|, then you have at least the
% following workflow:%
% \begin{quote}
% |pdflatex article|\\
% |python3 -m texgit.run article|\\
% |pdflatex article|
% \end{quote}
%
% All files are cached, usually in a directory called |__git__| in your
% document's direcotry.
% Unless you delete this directory, the cache will persist, even if you
% change your \LaTeX\ document and perform an entirely new compilation.
%
% Comprehensive examples are provided in \autoref{sec:examples}.
%
% \subsection{Installation}%
% \label{sec:installation}%
%
% \subsubsection{\LaTeX\ Package}%
% First, make sure that you have the |texgit.sty| either installed or inside
% your document's directory.
% For this, there are several options:%
% \begin{enumerate}%
% \item You can just download the file from
% \url{https://thomasweise.github.io/texgit_tex/texgit.sty} directly.
% You can now copy it into the folder of your document.
%%
% \item You can download |texgit.dtx|\tindex{texgit.dtx} and
% |texgit.ins|\tindex{texgit.ins} from
% \url{https://thomasweise.github.io/texgit_tex/texgit.dtx} and
% \url{https://thomasweise.github.io/texgit_tex/texgit.ins}.
% You can then execute
% \begin{quote}%
% |pdflatex texgit.ins|\tindex{pdflatex}
% \end{quote}%
% and you should get the style file |texgit.sty|\tindex{texgit.sty}.
% You can now copy it into the folder of your document.
%%
% \item Or you can download the |texgit.tds.zip| file from
% \url{https://thomasweise.github.io/texgit_tex/texgit.tds.zip} and
% unpack it into your \TeX\ tree.
% If you know what that is.
% \end{enumerate}%
%
% \subsubsection{Python Package}%
% \label{sec:texgitpy}%
% The Python\index{Python} package is available at
% \url{https://github.com/thomasWeise/texgit_py}\index{texgit\_py},
% \url{https://thomasweise.github.io/texgit_py}, and
% \url{https://pypi.org/project/texgit}.
% You can most easily install it from PyPI\index{PyPI} by doing
% \begin{quote}
% |pip install texgit|
% \end{quote}
%
% \subsubsection{git Executable}%
% Make sure that the |git| executable is available in the |PATH|.
% On Ubuntu Linux, you could install it via |sudo apt-get install git|, for example.
% |git| is needed because the primary use case of our package is to clone |git| repositories and include the files from these repositories (or to execute them and to include their output) into \LaTeX\ documents.
%
% \subsection{Loading the Package}%
% \label{sec:loadThePackage}%
% Load this package using%
% \begin{quote}
%   |\usepackage{texgit}|
% \end{quote}
% The package has no options or parameters.
%
% \emph{Notice:}~If you load this package, then you \emph{\textbf{must}} run the
% Python companion package inbetween |pdflatex| runs.
% Otherwise, there the second |pdflatex| run will abort with an error.
%
% Loading the package will automatically load the package
% |filecontents|\index{filecontents} as well,
% see \autoref{sec:implementation}.
%
% \subsection{Querying a File from a git Repository}%
% \label{sec:gitquery}%
% To query a file stored at path |path| inside from a |git|\tindex{git}
% repository available under URL |repositoryURL|, you would specify the
% command
% \begin{quote}
% |\gitLoad{id}{repositoryURL}{path}{postproc}|\tbindex{gitLoad}
% \end{quote}
% After this command is executed, a local path to the file becomes available
% in the command |\gitFile{id}|\tbindex{gitFile}.
% The full URL to the file in the |git| repository, including the current
% commit id, becomes available in the fully-expandable command
% |\gitUrl{id}|\tbindex{gitUrl}.
%
% Notice that you must choose unique values of |id| for every |\gitLoad|,
% |\gitExec|, and |\gitArg| invocation.
% You can invoke |\gitLoad| any number of times.
%
% The fourth parameter, |postproc|, which we may often leave empty, can
% specify an optional post-processing commend.
% If it is not left empty, this command will be executed in the
% shell\index{shell}.
% The contents of the file loaded from the |git| repository will be piped to
% the |stdin|\tindex{stdin} of the command.
% The |stdout|\tindex{stdout} of the command will be piped to a file and
% |\gitFile{id}|\tbindex{gitFile} will then contain the path to this file
% instead.
% For example, under Linux\index{Linux}, you could use the
% \href{https://www.man7.org/linux/man-pages/man1/head.1.html}{|head|}\index{head}
% command to return only the first 5~lines of a file as follows:
% \begin{quote}
% |\gitLoad{id}{repositoryURL}{path}{head -n 5}|\tbindex{gitLoad}
% \end{quote}
%
% \subsection{Executing a Command (optionally inside a git Repository}%
% \label{sec:gitexec}%
% Sometimes, we want to execute a program and fetch its standard output.
% \begin{quote}
% |\gitExec{id}{repositoryURL}{path}{theCommand}|\tbindex{gitExec}
% \end{quote}
% The most common use case of our package is that you want to execute a
% program which is part of a |git| repository.
% In this case, you would put the URL of the repository in |repositoryURL|
% and the relative path to the directory inside the repository in which the
% command should be invoked as |path|.
% If you want to invoke the command in the root folder of the repository,
% put |.| as |path|.
% The |theCommand| then holds the command line to be executed.
% \emph{Notice:}~You can also leave \emph{both} |repositoryURL| and
% |path| blank.
% In this case, the command is executed in the current folder.
% (The use case for this is to fetch the output of stuff like
% |python3 --version|.)
% Anyway, after this command is executed, a local path to the file with the
% captured standard output becomes available in the command
% |\gitFile{id}|\tbindex{gitFile}.
%
% Notice that you must choose unique values of |id| for every |\gitLoad|,
% |\gitExec|, and |\gitArg| invocation.
% You can invoke |\gitExec| any number of times.
%
% \subsection{Creating a File to be used as Argument}%
% \label{sec:gitexec}%
% Sometimes, we want to execute a program which requires a destination file.
% Let's say a program that creates a PDF figure.
% For this, we need to create an argument file on the fly.
% \begin{quote}
% |\gitArg{id1}{prefix}{suffix}|\tbindex{gitArg}\\
% |\gitExec{id2}{repositoryURL}{path}{theCommand (?id1?)}|\tbindex{gitExec}
% \end{quote}
% For this purpose, use the |\gitArg| command.
% This command takes a unique identifier~|id1|, a prefix, and a suffix as
% parameters.
% It will allocate a unique file path.
% This path can then be passed to a command to later invocation of |\gitExec|
% as parameter |(?id1?)| somewhere in its argument list.
% |(?id1?)| is resolved to the automatically generated file name before the
% actual command is invoked.
% After that, you can access the path via |\gitFile{id1}|\tbindex{gitFile}.
%
% \subsection{Executing the Python Package}
% \label{sec:pythonProgram}
% \begin{sloppypar}%
% During the first |pdflatex|\tindex{pdflatex} run, |\gitFile{id}| points to
% an empty dummy file
% (|\jobname.texgit.dummy|)\tbindex{jobname.texgit.dummy} and
% |\gitUrl{id}| points to |http://example.com|\index{http://example.com}.
% Both commands will only expand to useful information if the Python package
% |texgit| is applied to the project's |aux| file.
% This works very similar to \BibTeX.
% If the name of your \TeX\ file is |myfile.tex|, then you would execute
% \end{sloppypar}%
% \begin{quote}
% |python3 -m texgit.run myfile|
% \end{quote}
% More specifically, the Python package processes the |aux| files, so for a
% specific |aux|\tindex{aux} file |myfile.aux|, you could also do:
% \begin{quote}
% |python3 -m texgit.run myfile.aux|
% \end{quote}
%
% After this, in the next pass of |pdflatex|, |\gitFile{id}| and |\gitUrl{id}|
% will contain the right paths and URLs.
%
% \subsection{A Note on Python Virtual Environments}%
% The following only applies if you have installed this package inside a
% virtual environment.
% It also only applies in conjunction with version
% \href{https://pypi.org/project/texgit/0.8.17/}{0.8.17} or newer of
% the |texgit| Python package.
%
% If you are running this package inside a virtual environment, it is
% important that you create this environment using the |--copies| setting and
% \emph{not} using the (default) |--symlinks| parameter.
% In other words, you should have created the virtual environment as follows,
% where |venvDir| is the directory inside of which the virtual environment is
% created.
%
% \begin{center}
% |python3 -m venv --copies venvDir|
% \end{center}
%
% If you create the environment like this (and activated), then our package will
% automatically pick it up and use its Python interpreter for any invocation of
% |python3| or |python3.x| (where |x| is the minor version of the interpreter).
% If you use the |--symlinks| parameter to create the environment, then invocations
% of the Python interpreter from our package may instead result in the system's
% Python interpreter.
%
% \section{Provided Macros}\label{sec:providedMacros}%
%
% Here we discuss the macros that can directly be accessed by the user to
% make use of the functionality of the |texgit| package.
% The implementation of these macros is given in \autoref{sec:implementation}
% and several examples can be found in \autoref{sec:examples}.
%
% \subsection{gitLoad}
% \DescribeMacro{\gitLoad}%
% The macro |\gitLoad|\marg{id}\marg{repositoryURL}\marg{path}\marg{postProcessing}
% provides a local path to a file from a |git| repository.
% \begin{itemize}\sloppy%
%
% \item[\marg{id}] is the request ID chosen by the user.
% It must be unique over all requests made to |texgit|.
% Imagine it something like a label.
%
% \item[\marg{repositoryURL}] is the URL of the |git| repository.
% It could, e.g., be \url{https://github.com/thomasWeise/texgit\_tex} or
% \url{ssh://git@github.com/thomasWeise/texgit\_tex} or any other valid
% repository URL.
%
% \item[\marg{path}] is then the path to the file within the repository.
% This could be, for example, |latex/texgit.dtx|.%
%
% \item[\marg{postProcessing}] Can either be empty, in which case the
% repository is downloaded and the the local path to the file is returned.
% It can also be shell command, e.g., |head -n 5|.
% In this case, the contents of the file are piped to |stdin|\tindex{stdin} of
% the command and the text written to the |stdout|\tindex{stdout} by the
% command is stored in a file whose path is returned.
% \end{itemize}%
%
% You can access two results of this command via the following two commands:
% \begin{itemize}%
% \item[\texttt{{\textbackslash}gitFile\{id\}}] returns the path to the file
% that was loaded and/or post-processed.%
% \item[\texttt{{\textbackslash}gitUrl\{id\}}] returns the full URL to the
% file in the |git| repository
% online.
% This command works for \href{https://github.com}{GitHub}, but it may not
% provide the correct URL for other repository types.%
% \end{itemize}%
%
% \subsection{gitExec}
% \DescribeMacro{\gitExec}%
% The macro |\gitExec|\marg{id}\marg{repositoryURL}\marg{path}\marg{theCommand}
% provides a local path to a file containing the captured standard output
% of a command (that may have been executed inside a directory inside a
% |git| repository).
% \begin{itemize}\sloppy%
%
% \item[\marg{id}] is the request ID chosen by the user.
% It must be unique over all requests made to |texgit|.
% Imagine it something like a label.
%
% \item[\marg{repositoryURL}] is the URL of the |git| repository.
% It could, e.g., be \url{https://github.com/thomasWeise/texgit\_tex} or
% \url{ssh://git@github.com/thomasWeise/texgit\_tex} or any other valid
% repository URL.
% You can also leave this parameter empty if no |git| repository should
% be used.%
%
% \item[\marg{path}] is the path to a directory within the repository.
% This could be, for example, |latex| or |.|.
% If |path| is provided, then this will be the working directory where
% the command is executed.
% If you want to execute a command in the root directory of a |git|
% repository, you can put |.| here.
%
% \item[\marg{theCommand}] This is the command which should be executed.
% If |repositoryURL| and |path| are provided, then the repository will be
% downloaded and |path| will be resolved relative to the repository root
% directory. |theCommand| will then be executed in this directory.
% If neither |repositoryURL| nor |path| are provided, |theCommand| is
% executed in the current directory.
% Either way, its |stdout|\tindex{stdout} is captured in a file whose path
% is returned.
% \end{itemize}%
%
% After invoking this command its result can be obtained via the following
% command:%
% \begin{itemize}%
% \item[\texttt{{\textbackslash}gitFile\{id\}}] returns the path to the file
% in which the standard output is stored.%
% \end{itemize}%
%
% \subsection{gitArg}
% \DescribeMacro{\gitArg}%
% The macro |\gitArg|\marg{id}\marg{prefix}\marg{suffix} allows you to create
% a file whose name has a given |prefix| and |suffix|.
% This file can then be used in the argument list of the command invoked by
% |\gitExec| by writing |(?id?)|.
% Before that command is executed, |(?id?)| is resolved to the actual file
% name.
% This allows you to use commands that generate more structured output, say
% graphics.
% \begin{itemize}%
%
% \item[\marg{id}] is the request ID chosen by the user.
% It must be unique over all requests made to |texgit|.
% Imagine it something like a label.
%
% \item[\marg{prefix}] is a prefix for the file name to be generated.
% It can be empty.
%
% \item[\marg{suffix}] is the suffix for the file name to be generated.
% You can leave it empty.
% \end{itemize}%
%
% \subsection{gitFile}
% \DescribeMacro{\gitFile}%
% The macro |\gitFile|\marg{id} returns the path to the file with the
% contents of the |\gitLoad{id}...|, |\gitExec{id}....|, or
% |\gitArg{id}...| request using ID~|id|.
% During the first |pdflatex| pass, this will be the path to a dummy file.
% After the Python package has been applied to the |aux| file, then
% |\gitFile{id}| will point to the proper file during the next |pdflatex|
% pass.
%
%
% \subsection{gitUrl}
% \DescribeMacro{\gitUrl}%
% The macro |\gitUrl|\marg{id} returns the URL from which the file
% corresponding to the |\gitLoad{id}...| request was downloaded.
% This command is designed to work with GitHub.
% It will be the repository URL combine with the path of the file inside the
% repository and the commit has code.
% The Url thus points to the exact same version of the file that was
% downloaded (and optionally post-processed).
%
%
% \subsection{gitName}
% \DescribeMacro{\gitName}%
% The macro |\gitName|\marg{id} returns the name of the file
% corresponding to the |\gitLoad{id}...| request.
% This corresponds to the |basename| on unixoid file systems.
% In other words, only the local name of the file, without any
% other path components.
% This command makes sense only with files that are directly
% downloaded from |git| or filtered versions thereof, or \emph{maybe}
% argument files.
%
%
% \subsection{gitNameEsc}
% \DescribeMacro{\gitNameEsc}%
% The macro |\gitNameEsc|\marg{id} returns the \emph{escaped} name of the
% file corresponding to the |\gitLoad{id}...| request.
% This corresponds to the |basename| on unixoid file systems, as returned
% by |\gitName|, but with |_|, |$|, and | | escaped to |\_|, |\$|, and |~|
% respectively.
% This allows you to use the name of a file in plain \TeX, maybe like
% |\texttt{\gitNameEsc{myId}}|.
% This command makes sense only with files that are directly
% downloaded from |git| or filtered versions thereof, or \emph{maybe}
% argument files.
%
%
% \subsection{gitIf}
% \DescribeMacro{\gitIf}%
% The macro |\gitIf|\marg{id}{ifDone}{ifNotDone} executes the code provided as
% |ifDone| if the request with ID~|id| has already been executed and completed
% by the |texgit| Python package.
% Otherwise, it will execute the code provided as |ifNotDone|.
% As stated before, during the first |pdflatex| pass, all results of
% |\gitFile| and |\gitUrl| are dummies, either empty files or dummy URLs.
% However, some \LaTeX\ commands cannot deal with that.
% For example, say that we execute a program to generate a PDF graphic and
% store it in an argument file.
% During the first |pdflatex| run, this file is empty.
% If we try to |\includegraphics| it, this will crash.
% So we would place the |\includegraphics| into a |\gitIf| block and only
% execute it once the request is completed.
% After the first |pdflatex| run, we would execute the Python package
% |texgit|.
% This package will complete the request and the argument file will then be
% a proper graphic.
% In the second |pdflatex| pass, the |\includegraphics| code could be
% executed.
% An example for this is given in \autoref{sec:example_8}.
%
% \section{Examples}\label{sec:examples}%
%
% Here we provide a set of examples for the use of the package.
% Each example demonstrates another facet of the package and, at the same
% time, serves as test case.
% The first example given in \autoref{sec:example_1} is a Minimal Working
% Example, i.e., just provides the barest bones.
% It shows you how to import a single file from a |git| repository.
% The second example in \autoref{sec:example_2} shows you how to import
% multiple different files from different repositories (which equates to just
% using the same command multiple times) and how to use post-processors.
% The third example in \autoref{sec:example_3} shows how to create
% beautiful (to my standards) listings by including code from a |git|
% repository, post-processing it, and loading it as a |listing|.
% Finally, the fourth example in \autoref{sec:example_4} shows that you can
% also define macros for your favorite repository and post-processors to have
% a more convenient way to import files from |git|.
%
% \subsection{Minimal Working Example}%
% \label{sec:example_1}%
% This minimal working example shows how to load a file from a |git|
% repository and directly |\input|\expandafter\tbindex{input} its contents.
% The result can be seen in \autoref{ex:example_2:res}.
%
% \lstinputlisting[classoffset=2,morekeywords={A,B,C,D},breaklines=True,keywordstyle=\ttfamily\bfseries\color{blue!90!black},classoffset=0,float,label=ex:example_1,caption={A minimal working example for using the \texttt{texgit} package, rendered as \autoref{ex:example_1:res}. The contents of \texttt{dummy.tex} are shown in \autoref{ex:dummy_tex}.}]{examples/example_1.tex}%
%
% As you can see in \autoref{ex:example_1}, we first load the package
% |texgit| in line~2.
% Inside the document, we define a |git| request via the |\gitLoad| command.
% This command takes the ID of the request as first parameter.
% Here, we chose |R1|.
% Its second parameter is the URL of a |git| repository.
% In this case, this is \url{https://github.com/thomasWeise/texgit_tex},
% which happens to be the URL where you can find the repository of this
% package on GitHub.
% The third parameter is a path to the file in this repository relative to
% the repository root.
% In this case, this is the path to the file |examples/dummy.tex|, whose
% contents you can find in \autoref{ex:dummy_tex}.
%
% \lstinputlisting[float,label=ex:dummy_tex,caption={The contents of the file \texttt{dummy.tex} included from \texttt{git} in \autoref{ex:example_1}.}]{examples/dummy.tex}%
%
% The fourth parameter shall be ignored for now.
%
% After defining the request, we can now use two commands, |\gitFile{R1}| and
% |\gitUrl{R1}|.
% In this Minimal Working Example, we shall only consider the first one.
% This command expands to a local path of a file with the contents downloaded
% from the |git| repository.
%
% Well, during the first \LaTeX\ or |pdflatex| run, it just points to a dummy
% file with the name |\jobname.texgit.dummy|, where
% |\jobname|\tbindex{jobname} evaluates to the name of the main \LaTeX\
% document, say |article| for |article.tex|.
% At that point, the dummy file's content is a single space character followed
% by a newline.
%
% After the first |pdflatex| pass, you can apply the Python\index{Python}
% processor (see \autoref{sec:texgitpy}) as follows:
% \begin{quote}%
% |python3 -m texgit.run jobname|
% \end{quote}%
% Where |jobname| shall be replaced with the main file name, again |article|
% for |article.tex|, for instance.
%
% This command then downloads the file from |git| and puts it into a path that
% can locally accessed by \LaTeX.
% Usually, it will create a folder |__git__|\tindex{\_\_git\_\_} in your
% project's directory and place the file there.
%
% Anyway, during the second \LaTeX\ or |pdflatex| pass, |\gitFile{R1}| points
% to a valid file path with actual contents.
% By doing |\input{\gitFile{R1}}|\tbindex{input}, we here include this file
% (remember, its contents are given in \autoref{ex:dummy_tex}) as if it was
% part of our normal \LaTeX\ project.
% The result of this pass is shown in \autoref{ex:example_1:res}.
%
% If this example was stored as |example_1.tex|, then it could be built via
% \begin{quote}
% |pdflatex example_1|\\
% |python3 -m texgit.run example_1|\\
% |pdflatex example_1|
% \end{quote}
%
% \begin{figure}%
% \centering%
% \fbox{\includegraphics[page=1,width=0.9\linewidth,trim=1.65in 9.3in 1.5in 1.6in,clip]{examples/example_1.pdf}}%
% \caption{The rendered result of \autoref{ex:example_1} (with trimmed page margins and bottom).}%
% \label{ex:example_1:res}%
% \end{figure}%
%
% If we look back at the \autoref{ex:example_1} of our main file, you will
% notice the four blue marks {\bfseries\color{blue!90!black}A},
% {\bfseries\color{blue!90!black}B}, {\bfseries\color{blue!90!black}C}, and
% {\bfseries\color{blue!90!black}D}.
% These are just normal letters, colored and emphasized for your convenience.
% I put them there so that you can see where the action takes place.
% |\gitLoad| produces no output, so ``ABC'' come out next to each other.
% |\input{\gitFile}| between {\bfseries\color{blue!90!black}C} and
% {\bfseries\color{blue!90!black}D} loads and directly includes the example
% file, so this is where its content appear.
%
% One small interesting thing is that, since we directly |\input| the file,
% its contents are interpreted as \LaTeX\ code.
% This means that you could construct a document by inputting files from
% different |git| repositories.
%
% However, this is not the envisioned use case.
% The envisioned use case is to include source codes and snippets from source
% codes as listings.
% We will show how this could be done in the next example.
%
% \emph{Side note:}~Our Python companion package |texgit| downloads the
% |git| repositories into a folder called |__git__|\tindex{\_\_git\_\_} by
% default.
% If you do not delete the folder, the same repository will not be downloaded
% again but the downloaded copy will be used.
% This significantly increases speed and reduces bandwidth when applying the
% |texgit| command several times.
% \clearpage%
%
% \subsection{The Second Example: Multiple Files and Post-Processing}%
% \label{sec:example_2}%
% \FloatBarrier%
% In \autoref{ex:example_2} we, use |texgit| to download and present two
% different files from two different GitHub repositories.
% We also show how post-processing can work, once using the aforementioned
% simple |head -n 5| command available in the Linux shell and also by
% using the Python code formatting tool offered by the |texgit| Python
% package.
% The result can be seen in \autoref{ex:example_2:res}.
%
% \lstinputlisting[float,label=ex:example_2,caption={An example using the texgit package, rendered as \autoref{ex:example_2:res}.}]{examples/example_2.tex}%
% \FloatBarrier%
%
% The file |example_2.tex| shown in \autoref{ex:example_2} begins by loading
% our |texgit| package as well as package |verbatim|, which is later used
% to display the included files.
% The document creates three sections, each of which is used to display one
% imported file.
%
% The first section loads one Python source file from the Python
% package \href{https://thomasweise.github.io/pycommons}{pycommons}.
% The sources of this package are available in the GitHub repository
% \url{https://github.com/thomasWeise/pycommons}.
% We download the file |pycommons/io/console.py|, which is just a small
% utility for printing log strings to the output together with a time mark.
% The full request, with the ID |R2|, contains these two components.
%
% Issuing this request will set the command |\gitFile{R2}| to the local file
% containing the downloaded contents of |pycommons/io/console.py| from the
% repository \url{https://github.com/thomasWeise/pycommons}.
% The command |\gitUrl{R2}| will expand to the URL pointing to the downloaded
% \emph{version} of the file in the original repository.
% This command, at the present time, is only really valid for GitHub.
% It builds a URL relative to the original repository based on the commit ID
% that was valid when the file was downloaded from the repository.
% Therefore, the URL then points to the \emph{exact same} contents that were
% put into the file.
% Anyway, the file contents and the generated URL are displayed in
% \autoref{ex:example_2:1}.
%
% The second section of the example document queries the same file again.
% However, this time, the fourth parameter of |\gitLoad| is specified.
% If the fourth parameter is left blank, the downloaded file will be provided
% as-is.
% However, especially if we would like to include some snippets of a more
% complex source file, we sometimes do not want to have the complete original
% contents.
% In this case, we can specify a post-processing command as third parameter.
% This command will be executed in the shell\index{shell}
% The contents of the downloaded file will then be piped into the
% |stdin|\tindex{stdin} of the command and everything that the command writes
% to its |stdout|\tindex{stdout} will be collected in a file.
% We use ID |R2| for this request.
% |\gitFile{R2}| then returns the path to that file.
%
% Since you can provide arbitrary commands as post-processors, this allows you
% to do, well, arbitrary post-processing.
% This could include re-formatting of code or selecting only specific lines
% from the file.
% The command can have arguments, separated by spaces, allowing you to pass
% information such as line indices or other instructions to your
% post-processing command.
%
% In the example, we use the standard Linux command |head -n 5|, which writes
% the first five lines that were written to its |stdin| to its |stdout|.
%
% The resulting output in \autoref{ex:example_2:2} looks thus similar to
% \autoref{ex:example_2:1}, but only imports ths first five lines from the
% downloaded file.
%
% If this example was stored as |example_2.tex|, then it could be built via
% \begin{quote}
% |pdflatex example_2|\\
% |python3 -m texgit.run example_2|\\
% |pdflatex example_2|
% \end{quote}
%
% \emph{Side note:}~Such post-processing steps are cached by the Python
% companion package |texgit| in the |__git__|\tindex{\_\_git\_\_} folder
% as well.
%
% Finally, in the third section, of \autoref{ex:example_2}, we import a file
% from the sources of our Python package for metaheuristic optimization
% (\href{https://thomasweise.github.io/moptipy}{moptipy}).
% The sources of this package are located on GitHub at
% \url{https://github.com/thomasWeise/moptipy}.
% We download the file |moptipy/api/encoding.py|, which offers a convenient
% API for implementing an \emph{encoding} which translates from the search to
% the solution space (but that would lead too far here).
% Either way, this is a file that has lots of content.
% So we want to select certain contents while ignoring other.
% We also remove all Python type hints and all comments from the source and
% then reformat it.
%
% Luckily, our |texgit| Python package also offers a Python code formatter,
% namely the executable module \href{https://thomasweise.github.io/texgit_py/texgit.formatters.html#module-texgit.formatters.python}{texgit.formatters.python}.
% This module takes a set of parameters such as limiting |labels| that denote
% the start and end of code snippets (in this case, the label ``|book|'') to
% include as well |args| telling the system which part of the ``omittable''
% code to preserve (in this case, preserve |doc|strings and delete everything
% else that is non-essential).
% If you are interested in such post-processing, feel invited to check out
% the \href{https://thomasweise.github.io/texgit_py}{documentation} of the
% Python companion package at \url{https://thomasweise.github.io/texgit_py}.
% Either way, the file is downloaded, piped through this post-processor, and
% the result is included as shown in \autoref{ex:example_2:3}.
%
% \begin{figure}%
% \centering%
% \strut\hfill\strut%
% \subcaptionbox{Page 1 of the pdf compiled from \autoref{ex:example_2}.\label{ex:example_2:1}%
% }{%
% \fbox{\includegraphics[page=1,width=0.455\linewidth,trim=1.65in 5in 1.5in 1.5in,clip]{examples/example_2.pdf}}%
% }%
% \strut\hfill\strut\hfill\strut%
% \subcaptionbox{Page 2 of the pdf compiled from \autoref{ex:example_2}.\label{ex:example_2:2}%
% }{%
% \fbox{\includegraphics[page=2,width=0.455\linewidth,trim=1.65in 5in 1.5in 1.5in,clip]{examples/example_2.pdf}}%
% }%
% \strut\hfill\strut%
% \\%
% \strut\hfill\strut%
% \subcaptionbox{Page 3 of the pdf compiled from \autoref{ex:example_2}.\label{ex:example_2:3}%
% }{\parbox{\linewidth}{\centering%
% \fbox{\includegraphics[page=3,width=0.455\linewidth,trim=1.65in 5in 1.5in 1.5in,clip]{examples/example_2.pdf}}%
% }}%
% \strut\hfill\strut%
%
% \caption{The rendered result of \autoref{ex:example_2} (with trimmed page margins and bottoms).}%
% \label{ex:example_2:res}%
% \end{figure}%
%
% \FloatBarrier%
%
%
% \subsection{The Third Example: Using the \texttt{listings} Package}%
% \label{sec:example_3}%
% \FloatBarrier%
% As third example, let us show the interaction with the package |listings|.
% This is not much different from using the package |verbatim| in the second
% example above.
% I just wanted to show you how it looks like.
% Also, I wanted to show the intended use of |\gitUrl|:
% You can use it to put some small ``(src)'' link in the listing's caption.
% This way, you can create teaching material where every listing is linked to
% the correct version of source code online without splattering long URLs into
% your text.
% Anyway.
% The source code of the third example is given in \autoref{ex:example_3} and
% the compiled result as \autoref{ex:example_3:res}.
%
% \lstinputlisting[float,label=ex:example_3,caption={An example using the \texttt{listings} package, rendered as \autoref{ex:example_3:res}.}]{examples/example_3.tex}%
%
% \begin{figure}%
% \centering%
% \fbox{\includegraphics[page=1,width=0.9\linewidth,trim=1.4in 5.6in 1.5in 1.6in,clip]{examples/example_3.pdf}}%
% \caption{The rendered result of \autoref{ex:example_3} (with trimmed page margins and bottom).}%
% \label{ex:example_3:res}%
% \end{figure}%
%
% If this example was stored as |example_3.tex|, then it could be built via
% \begin{quote}
% |pdflatex example_3|\\
% |python3 -m texgit.run example_3|\\
% |pdflatex example_3|
% \end{quote}
%
% \emph{Side note:}~If you actually check the \href{https://thomasweise.github.io/moptipy/_modules/moptipy/algorithms/so/rls.html}{source code}
% of the RLS algorithm, which is linked to by the ``(\href{https://thomasweise.github.io/moptipy/_modules/moptipy/algorithms/so/rls.html}{src})''
% in the caption of the example and that is displayed in the example, you will
% find that it actually uses Python type hints.
% It also has a comprehensive doc-string and is commented well.
% In source code of a real project, we do want this.
% In a listing in a book, we do not.
% The post-processor command
% \begin{quote}
% |python3 -m texgit.formatters.python --labels book|
% \end{quote}%
% only keeps the code between the labels ``|# start book|'' and
% ``|# end book|.''
% It also removes all non-essential stuff such as type hints, comments, and
% the doc-string.
% Then it re-formats the code to save space.
% Again, check out the documentation of our |texgit| Python companion
% package at \url{https://thomasweise.github.io/texgit_py}.
% This is the main intended use case of our package:
% Be able to have nicely documented ``real'' code and to use parts of it
% in teaching materials.
%
% \FloatBarrier%
%
%
% \subsection{The Fourth Example: Using Git Commands in Macros}%
% \label{sec:example_4}%
% \FloatBarrier%
%
% \lstinputlisting[float,label=ex:example_4,caption={An example using commands from the texgit package in macros, rendered as \autoref{ex:example_4:res}.}]{examples/example_4.tex}%
%
% The goal of the fourth example is to show that we can also put the commands
% from our |texgit| package into \LaTeX\ macros.
% We define a new command |\moptipySrc| with three parameters.
% \href{https://thomasweise.github.io/moptpiy}{moptipy} is a Python package
% that implements lots of metaheuristic algorithms.
% We could want to load several files from such a repository
% \url{https://github.com/thomasWeise/moptipy} and post-process and display
% them all in the same way.
% Then, it would be annoying to always do |\gitLoad|, |\lstinputlisting|,
% and spell out the post-processor each time.
% So we put all of this into a single command whose first argument is the
% label to put for the listing, whose second command is the caption to use,
% and whose third command is the path relative to the folder ``moptipy'' in
% the |git| repository.
% We also use this first parameter, the label, as ID for our |texgit|
% commands.
% In \autoref{ex:example_4}, we can then simply call |\moptipySrc| and it will
% do the whole process of loading a file from the right repository,
% post-processing it, putting a floating listing, and even putting a small
% ``(\href{https://thomasweise.github.io/moptpiy}{src})'' into the caption of
% the listing.
% The results are shown in \autoref{ex:example_4:res}% and can be obtained via
% \begin{quote}
% |pdflatex example_4|\\
% |python3 -m texgit.run example_4|\\
% |pdflatex example_4|
% \end{quote}
% (if the example code from \autoref{ex:example_4} was stored in a file called
% |example_4.tex|, that is.)
%
% \begin{figure}%
% \centering%
% \strut\hfill\strut%
% \subcaptionbox{Page 1 of the pdf compiled from \autoref{ex:example_4}.\label{ex:example_4:1}%
% }{%
% \fbox{\includegraphics[page=1,width=0.455\linewidth,trim=1.65in 6.2in 1.5in 1.8in,clip]{examples/example_4.pdf}}%
% }%
% \strut\hfill\strut\hfill\strut%
% \subcaptionbox{Page 2 of the pdf compiled from \autoref{ex:example_4}.\label{ex:example_4:2}%
% }{%
% \fbox{\includegraphics[page=2,width=0.455\linewidth,trim=1.65in 3.7in 1.5in 3.7in,clip]{examples/example_4.pdf}}%
% }%
% \strut\hfill\strut%
% \caption{The rendered result of \autoref{ex:example_4} (with trimmed page margins and bottoms).}%
% \label{ex:example_4:res}%
% \end{figure}%
%
% \FloatBarrier%
%
%
% \subsection{The Fifth Example: Capturing the Output of a Program}%
% \label{sec:example_5}%
% \FloatBarrier%
%
% \lstinputlisting[float,label=ex:example_5,caption={An example of capturing the output of a program, rendered as \autoref{ex:example_5:res}.}]{examples/example_5.tex}%
%
% The goal of the fifth example is to show that we can capture the output of
% a program.
% In \autoref{ex:example_5}, we just invoke |python3 --version| and capture
% the output in a file.
% We then load this file as listing.
% The results are shown in \autoref{ex:example_5:res}% and can be obtained via
% \begin{quote}
% |pdflatex example_5|\\
% |python3 -m texgit.run example_5|\\
% |pdflatex example_5|
% \end{quote}
% (if the example code from \autoref{ex:example_5} was stored in a file called
% |example_5.tex|, that is.)
%
% \begin{figure}%
% \centering%
% \fbox{\includegraphics[page=1,width=0.9\linewidth,trim=1.65in 7.9in 1.5in 1.8in,clip]{examples/example_5.pdf}}%
% \caption{The rendered result of \autoref{ex:example_5} (with trimmed page margins and bottoms).}%
% \label{ex:example_5:res}%
% \end{figure}%
%
%
% \FloatBarrier%
%
%
% \subsection{The Sixth Example: Capturing the Output of a Program Executed Inside a git Repository}%
% \label{sec:example_6}%
% \FloatBarrier%
%
% \lstinputlisting[float,label=ex:example_6,breaklines=True,caption={An example of capturing the output of a program executed inside a git repository, rendered as \autoref{ex:example_6:res}.}]{examples/example_6.tex}%
%
% The goal of the sixth example is to show that we can capture the output of
% a program -- but this time we execute it inside a |git| repository.
% In \autoref{ex:example_6}, we invoke a program which is part of the examples
% suite of the \href{https://github.com/thomasWeise/pycommons}{pycommons}
% utility package.
% We capture its standard output in a file.
% We then load this file as listing.
% The results are shown in \autoref{ex:example_6:res}% and can be obtained via
% \begin{quote}
% |pdflatex example_6|\\
% |python3 -m texgit.run example_6|\\
% |pdflatex example_6|
% \end{quote}
% (if the example code from \autoref{ex:example_6} was stored in a file called
% |example_6.tex|, that is.)
%
% \begin{figure}%
% \centering%
% \fbox{\includegraphics[page=1,width=0.9\linewidth,trim=1.65in 7.5in 1.5in 1.8in,clip]{examples/example_6.pdf}}%
% \caption{The rendered result of \autoref{ex:example_6} (with trimmed page margins and bottoms).}%
% \label{ex:example_6:res}%
% \end{figure}%
%
%
% \FloatBarrier%
%
%
% \subsection{The Seventh Example: Capturing the Output of Multiple Programs Executed Inside Different git Repositories}%
% \label{sec:example_7}%
% \FloatBarrier%
%
% \lstinputlisting[float,label=ex:example_7,breaklines=True,caption={An example of capturing the output of three programs executed inside different git repositories, rendered as \autoref{ex:example_7:res}.}]{examples/example_7.tex}%
%
% The goal of the seventh example is to show that we can capture the output of
% multiple programs from inside different |git| repositories.
% In \autoref{ex:example_7}, we invoke the same program as in \autoref{ex:example_6} and of two
% programs which are part of the examples suite of the
% \href{https://github.com/thomasWeise/programmingWithPython}{Programming with Python}
% book. The examples can be found in the repository
% \url{https://github.com/thomasWeise/programmingWithPythonCode}, whereas the book can be downloaded
% from \url{https://github.com/thomasWeise/programmingWithPython}.
% We capture the standard output of both programs in three files.
% We then load these file as listings.
% The results are shown in \autoref{ex:example_7:res} and can be obtained via
% \begin{quote}
% |pdflatex example_7|\\
% |python3 -m texgit.run example_7|\\
% |pdflatex example_7|
% \end{quote}
% (if the example code from \autoref{ex:example_7} was stored in a file called
% |example_7.tex|, that is.)
%
% \begin{figure}%
% \centering%
% \fbox{\includegraphics[page=1,width=0.9\linewidth,trim=1.65in 6.1in 1.5in 1.5in,clip]{examples/example_7.pdf}}
% \caption{The rendered result of \autoref{ex:example_7} (with trimmed page margins and bottoms).}%
% \label{ex:example_7:res}%
% \end{figure}%
%
% \afterpage{\clearpage}%
%
%
% \FloatBarrier%
%
%
% \subsection{The Eight Example: Using Argument Files}%
% \label{sec:example_8}%
% \FloatBarrier%
%
% \lstinputlisting[float,label=ex:example_8,caption={An example of running a script residing in git and passing an automatically generated file path to it as input, rendered as \autoref{ex:example_8:res}.}]{examples/example_8.tex}%
% \lstinputlisting[float,language=Python,label=py:make_pdf,caption={The Python script invoked in \autoref{ex:example_8} to create a graphic and store it in an argument file.}]{examples/make_pdf.py}%
%
% \begin{sloppypar}%
% The goal of the eighth example is to show how to use argument files.
% In \autoref{ex:example_8}, we first create an argument file via
% |\gitArg{R13}{test}{.pdf}|.
% This file will have the name prefix |test| and the suffix |.pdf|.
% It will be available under the ID~|R13|.
% The command |\gitExec{}{...}{...}{python3 make_pdf.py (?R13?)}| invokes
% the Python program |make_pdf.py| residing in a certain directory of a |git|
% repository (actually, our repository here).
% We leave the first parameter, the |id|, empty, because we are not interested
% in the output of this program.
% The agument |(?R13?)| passed to the program will be resolved to the argument
% file path before invoking the program.
% The program is illustrated in \autoref{py:make_pdf}.
% It generates a PDF graphic and stores it in the file that it received as
% argument.
% We can then include this graphic using |\includegraphics| and passing in
% |\gitFile{R13}|.
% \end{sloppypar}%
%
% However, |\gitFile{R13}| will only be valid after we applied the Python
% |texgit| command \emph{after} the first |pdflatex| pass.
% Therefore, during the first |pdflatex| pass, this file will be empty.
% |\includegraphics| would fail.
% Therefore, we protect this code to be only executed once the file has been
% filled with data.
% This is done by placing it into the
% |\gitIf{R13}{<exec if done>}{<otherwise>}| block.
%
% The results are shown in \autoref{ex:example_8:res} and can be obtained via
% \begin{quote}
% |pdflatex example_8|\\
% |python3 -m texgit.run example_8|\\
% |pdflatex example_8|
% \end{quote}
% (if the example code from \autoref{ex:example_8} was stored in a file called
% |example_8.tex|, that is.)
%
% \begin{figure}%
% \centering%
% \fbox{\includegraphics[page=1,width=0.7\linewidth,trim=1in 1in 1in 1in,clip]{examples/example_8.pdf}}
% \caption{The rendered result of \autoref{ex:example_8} (with trimmed page margins and bottoms).}%
% \label{ex:example_8:res}%
% \end{figure}%
%
% \afterpage{\clearpage}%
%
% \section{A Note on LaTeX Floating Environments}
% \begin{sloppypar}%
% \LaTeX\ offers us floating environments for laying out tables and figures.
% Such environments, e.g., |\begin{figure}...\end{figure}| or
% |\begin{table}...\end{table}| are automatically arranged by \LaTeX.
% This means that their order can change depending how they layout engine
% determines where to put them.
% This means that, if you put |\gitLoad|, |\gitArg|, or |\gitExec| commands
% into such environments, their execution order can change.
% It is therefore recommended to put these commands always outside of these
% environments.
% Then the execution order is always the same and clear.
% \end{sloppypar}%
% \section{Usage in GitHub Actions}
% It is possible to use |texgit| in GitHub actions.
% For this purpose, you would simply copy the |texgit.sty| file into the directory of your \LaTeX\ sources.
% You can use \href{https://github.com/xu-cheng/texlive-action}{xu-cheng/texlive-action}, which provides TexLive, to run several |pdflatex| / |texgit| cycles.
% Then, you could deploy the book that was built using \href{https://github.com/JamesIves/github-pages-deploy-action}{JamesIves/github-pages-deploy-action} to a GitHub page.
% This way, whenever you commit changes \LaTeX\ sources residing in a |git| repository, your book would be re-built.
% The projects below do it this way.
%
% \section{Projects using texgit}
% |texgit| was developed specifically for teaching material book projects.
% There are several book projects that use it.
% \begin{itemize}%
% \item \emph{Programming with Python}, available at \url{https://thomasWeise.github.io/programmingWithPython} is a book and a set of slides introducing the reader to Python programming.
% It is enriched with many source code examples, which reside in the repository \url{https://github.com/thomasWeise/programmingWithPythonCode}.
% These examples are automatically included and executed in the book building process via |texgit|.
% \item \emph{Databases}, available at \url{https://thomasWeise.github.io/databases} is a book and a set of slides introducing the reader to relational databases.
% It is enriched with many source code examples, which reside in the repository \url{https://github.com/thomasWeise/databasesCode}.
% These examples are automatically included and executed in the book building process via |texgit|.
% Actually, they are |sql| scripts that run on a PostgreSQL database management system.
% With |texgit|, you can run complex scripts residing in |git| repositories.
% It is possible to interact with a database server without problems.
%
% \end{itemize}%
%
% \clearpage%
%
% \StopEventually{}
%
% \section{Implementation}%
% \label{sec:implementation}%
% The names of all internal elements of the package are prefixed
% with |@texgit@|.
% This naming convention should prevent any name clashes with other packages.
%
% Our |texgit| package requires only one other package:%
% \begin{enumerate}%
% \item |filecontents|\index{filecontents}~\cite{P2023TFP} is used to allow us
% to generate the dummy file on the fly.
% This package is obsolete for the most recent \LaTeX\ version, where it
% simply does nothing, but may help us to get our package to work on older
% systems.
% \end{enumerate}%
%    \begin{macrocode}
\RequirePackage{filecontents}%  Allow us to create the dummy file.
%    \end{macrocode}
%
%    \begin{macrocode}
%
% This is the path to the dummy file.
% The dummy file is created directly below.
% The dummy file is referenced by all invocations of |\gitFile| until the
% Python package has been applied to the |.aux| file and has loaded the
% actual files.
\xdef\@texgit@dummyPath{\jobname.texgit.dummy}%  the dummy file
%
% Create the dummy file that replaces files before they are loaded.
% This file only has one line with one single space.
\expandafter\begin\expandafter{filecontents*}{\@texgit@dummyPath}

\end{filecontents*}
%
%% We need to make sure that the texgit postprocessor is actually
%% properly applied.
%% If you load our package, then if the postprocessor is not applied
%% before the second pdflatex pass, we make sure that pdflatex
%% crashes with an error.
\protected\gdef\@texgit@needsTexgitPass{%
\errmessage{texgit: You must run the Python companion package of texgit %
*before* doing the second pdflatex pass. %
You can do this by invoking 'python3 -m texgit.run \jobname'. %
You can obtain this Python package via 'pip install texgit'. %
Check the documentation at %
https://thomasweise.github.io/texgit_tex/texgit.pdf%
}%
%% force quit
\batchmode\read-1 to \foo%
}%
%% Make sure that the texgit postprocessor is actually applied.
\AtEndDocument{%
\let\@texgit@needsTexgitPass\relax%
\immediate\write\@mainaux{\noexpand\@texgit@needsTexgitPass}}%
%
% The dummy URL that is returned by |\gitURL| unless a proper URL is
% available.
\xdef\@texgit@dummyUrl{https://example.com}%  the dummy URL
%
% This command does nothing and is just a placeholder in the |aux| files.
\gdef\@texgit@gitFile#1#2#3#4{}%
% This command as well.
\gdef\@texgit@process#1#2#3#4{}%
% This command as well.
\gdef\@texgit@argFile#1#2#3{}%
%    \end{macrocode}
%
% \begin{macro}{\gitLoad}
% The macro |\gitLoad|\marg{id}\marg{repositoryURL}\marg{path}\marg{postProcessing}
% defines a query to a |git| repository.
% The query is stored in the |aux| file of the project and carried out by the
% Python companion package (see \autoref{sec:pythonProgram}).
% The results of this macro will become available via the two other macros
% |\gitFile{id}| and |\gitUrl{id}|.
% During the first \LaTeX\ build, these macros will return a path to a dummy
% file which only has a single space character in it followed by a newline and
% the dummy URL \url{https://example.com}, respectively.
% As said, |\gitLoad| will store all information in the |aux| file, which then
% permits the |texgit| Python package to download (and optionally
% post-process) the actual file.
% In the second round of \LaTeX\ building, |\gitFile{id}| and |\gitUrl{id}| will
% then return the local path to that downloaded file and the actual URL,
% respectively.%
%
% \begin{itemize}\sloppy%
% \item[\marg{id}] the uniquely chosen ID for this request.
% Imagine this being something like a label.
%
% \item[\marg{repositoryURL}] is the URL of the |git| repository.
% It could, e.g., be \url{https://github.com/thomasWeise/texgit\_tex} or
% \url{ssh://git@github.com/thomasWeise/texgit\_tex} or any other valid
% repository URL.%
%
% \item[\marg{path}] is then the path to the file within the repository.
% This could be, for example, |latex/texgit.dtx|.%
%
% \item[\marg{postProcessing}] Can either be empty, in which case the
% repository is downloaded and the the local path to the file is
% returned.
% It can also be shell command, e.g., |head -n 5|.
% In this case, the contents of the file are piped to |stdin| of the command
% and the text written to the |stdout| by the command is stored in a file
% whose path is returned.
% \end{itemize}%
%    \begin{macrocode}
%%
%% Define a query to load and post-process a file from a |git| repository.
%% #1 is the request ID
%% #2 is the repository URL
%% #3 is the path to the file inside the repository
%% #4 is a command through which the file contents should be piped
%%%   (leave #4 empty to use the file as-is)
\protected\gdef\gitLoad#1#2#3#4{%
\edef\@texgit@pA{#1}%  fully expand the request ID
\edef\@texgit@pB{#2}%  fully expand the repository URL
\edef\@texgit@pC{#3}%  fully expand the path into the repository
\edef\@texgit@pD{#4}%  fully expand the (optional) shell command
% Write the parameters to the aux file.
\immediate\write\@mainaux{%
\noexpand\@texgit@gitFile{\@texgit@pA}{\@texgit@pB}{%
\@texgit@pC}{\@texgit@pD}}%
}%
%    \end{macrocode}
% \end{macro}
%
% \begin{macro}{\gitExec}
% The macro |\gitExec|\marg{id}\marg{repositoryURL}\marg{path}\marg{theCommand}
% defines a command to be executed either inside a |git| repository or
% in the current directory.
% The query is stored in the |aux| file of the project and carried out by the
% Python companion package (see \autoref{sec:pythonProgram}).
% The results of this macro will become available via the macro
% |\gitFile{id}|.
% During the first \LaTeX\ build, this macro will return a path to a dummy
% file which only has a single space character in it followed by a newline.
% As said, |\gitExec| will store all information in the |aux| file, which then
% permits the |texgit| Python package to download (and optionally
% post-process) the actual file.
% In the second round of \LaTeX\ building, |\gitFile{id}| will then return the
% local path to the file with the standard output of the executed
% command.%
%
% \begin{itemize}\sloppy%
% \item[\marg{id}] the uniquely chosen ID for this request.
% Imagine this being something like a label.
%
% \item[\marg{repositoryURL}] is the URL of the |git| repository.
% It could, e.g., be \url{https://github.com/thomasWeise/texgit\_tex} or
% \url{ssh://git@github.com/thomasWeise/texgit\_tex} or any other valid
% repository URL.%
% You can leave this argument empty if you want to execute the command in the
% current directory.
%
% \item[\marg{path}] is then the path to the directory within the repository.
% This could be, for example, |latex|.
% The command is executed at this directory.
% Use |.| for the repository root.
% Leave this empty if no repository is used.
%
% \item[\marg{theCommand}] The command line to be executed.
% It can also be shell command, e.g., |python3 --version|.
% The standard output produced by this command is captured as file.
% \end{itemize}%
%    \begin{macrocode}
\newcount\@texgit@counter%  The counter for the dummy arguments of \gitExec
\@texgit@counter0\relax%    We start the counter at 0.
%%
%% Define a query to execute a command, optionally in a |git| repository.
%% #1 is the request ID, leave empty if you do not want to access the output
%% #2 is the repository URL, or empty if no repository is needed
%% #3 is the path to a directory inside the repository or empty
%% #4 is a command to be executed
\protected\gdef\gitExec#1#2#3#4{%
\edef\@texgit@pA{#1}%  the request ID
\ifx\@texgit@pA\empty\relax\edef\@texgit@pA{%
@texgit@dummy@\the\@texgit@counter}%
\global\advance\@texgit@counter by 1% step the counter for the next dummy
\fi%
\edef\@texgit@pB{#2}%  fully expand the repository URL, or empty
\edef\@texgit@pC{#3}%  fully expand the path into the repository, or empty
\edef\@texgit@pD{#4}%  fully expand the (optional) shell command
% Write the parameters to the aux file.
\immediate\write\@mainaux{%
\noexpand\@texgit@process{\@texgit@pA}{\@texgit@pB}{%
\@texgit@pC}{\@texgit@pD}}%
}%
%    \end{macrocode}
% \end{macro}
%
% \begin{macro}{\gitArg}
% The macro |\gitArg|\marg{id}\marg{prefix}\marg{suffix}
% allocates a unique file name with the given |prefix| and |suffix|
% that can be passed as argument in invocations of |\gitExec|.
% The results of this macro will become available via the macro
% |\gitFile{id}|.
% During the first \LaTeX\ build, this macro will return a path to a dummy
% file which only has a single space character in it followed by a newline.
% As said, |\gitExec| will store all information in the |aux| file, which then
% permits the |texgit| Python package to download (and optionally
% post-process) the actual file.
% In the second round of \LaTeX\ building, |\gitFile{id}| will then return the
% local path to the file with the standard output of the executed
% command.%
%
% \begin{itemize}\sloppy%
% \item[\marg{id}] the uniquely chosen ID for this request.
% Imagine this being something like a label.
%
% \item[\marg{prefix}] is the optional prefix, which can be left empty.
%
% \item[\marg{suffix}] is the optional suffix, which can be left empty.
%
% \end{itemize}%
%    \begin{macrocode}
%%
%% Define a query to allocate a file.
%% #1 is the request ID
%% #2 is the prefix
%% #3 is the suffix
\protected\gdef\gitArg#1#2#3{%
\edef\@texgit@pA{#1}%  the request ID
\edef\@texgit@pB{#2}%  fully expand the prefix
\edef\@texgit@pC{#3}%  fully expand the suffix
% Write the parameters to the aux file.
\immediate\write\@mainaux{%
\noexpand\@texgit@argFile{\@texgit@pA}{\@texgit@pB}{%
\@texgit@pC}}%
}%
%    \end{macrocode}
% \end{macro}
%
%
% \begin{macro}{\gitIf}
% The macro |\gitIf|\marg{id}\marg{ifDone}\marg{ifNotDone} executes |ifDone|
% if the request under the given |id| has already been executed, otherwise
% executes |ifNotDone|.
%
% \begin{itemize}%
% \item[\marg{id}] the request ID.
% \item[\marg{ifDone}] executed if the request with the given ID has already
% been executed.
% \item[\marg{ifNotDone}] executed if the request with the given ID has not
% yet been executed.
% \end{itemize}%
%    \begin{macrocode}
%%
%% Conditionally execute code depending on whether the given request was
%% already executed.
%% #1 is the request ID
%% #2 the code to execute if the request of the given ID has been executed.
%% #3 the code to execute if the request of the given ID has NOT been
%%    executed.
\gdef\gitIf#1#2#3{%
\expandafter\ifx\csname @texgit@path@#1\endcsname\relax%
\expandafter\@firstoftwo\else\expandafter\@secondoftwo\fi{#3}{#2}%
}%
%    \end{macrocode}
% \end{macro}
%
%
% \begin{macro}{\gitFile}
% The macro |\gitFile|\marg{id} get the file path associated with the request
% of ID~|id|.
%
% \begin{itemize}%
% \item[\marg{id}] the request ID.
% yet been executed.
% \end{itemize}%
%    \begin{macrocode}
%%
%% Get the file path associated with the request ID.
%% #1 is the request ID
\gdef\gitFile#1{%
\expandafter\ifx\csname @texgit@path@#1\endcsname\relax%
\@texgit@dummyPath\else\csname @texgit@path@#1\endcsname\fi%
}%
%    \end{macrocode}
% \end{macro}
%
%
% \begin{macro}{\gitUrl}
% The macro |\gitUrl|\marg{id} get the URL associated with the request of
% ID~|id|.
%
% \begin{itemize}%
% \item[\marg{id}] the request ID.
% yet been executed.
% \end{itemize}%
%    \begin{macrocode}
%%
%% Get the URL associated with the request ID.
%% #1 is the request ID
\gdef\gitUrl#1{%
\expandafter\ifx\csname @texgit@url@#1\endcsname\relax%
\@texgit@dummyUrl\else\csname @texgit@url@#1\endcsname\fi%
}%
%    \end{macrocode}
% \end{macro}
%
%
% \begin{macro}{\gitName}
% The macro |\gitName|\marg{id} gets the base name of the path associated
% with the request of ID~|id|.
% Like |\gitUrl|, this makes only sense if the file is hosted in a |git|
% repository.
% If the file contains the data generated by a process, then this command
% returns an empty string.
%
% \begin{itemize}%
% \item[\marg{id}] the request ID.
% yet been executed.
% \end{itemize}%
%    \begin{macrocode}
%%
%% Get the base name of the path associated with the request ID.
%% #1 is the request ID
\gdef\gitName#1{%
\expandafter\ifx\csname @texgit@name@#1\endcsname\relax%
\else\csname @texgit@name@#1\endcsname\fi%
}%
%    \end{macrocode}
% \end{macro}
%
%
% \begin{macro}{\gitNameEsc}
% The macro |\gitNameEsc|\marg{id} gets the escaped base name of the path
% associated with the request of ID~|id|.
% Sometimes, file names may contain characters like |_| or |$|.
% This means that you could not print the result of |\getFileName| in plain
% \TeX, because it would cause an error during compilation.
% |\getNameEsc| escapes |_|, |$|, and | | to |\_|, |\$|, and |~|,
% respectively.
% Therefore, its result can be typeset like normal text, at least in
% |\texttt{...}|.
% Like |\gitUrl|, this makes only sense if the file is hosted in a |git|
% repository.
% If the file contains the data generated by a process, then this command
% returns an empty string.
%
% \begin{itemize}%
% \item[\marg{id}] the request ID.
% yet been executed.
% \end{itemize}%
%    \begin{macrocode}
%%
%% Get the base name of the path associated with the request ID.
%% #1 is the request ID
\gdef\gitNameEsc#1{%
\expandafter\ifx\csname @texgit@escName@#1\endcsname\relax%
\else\csname @texgit@escName@#1\endcsname\fi%
}%
%    \end{macrocode}
% \end{macro}
%
% \begin{thebibliography}{10}
% \providecommand{\natexlab}[1]{#1}
% \providecommand{\url}[1]{\texttt{#1}}
% \expandafter\ifx\csname urlstyle\endcsname\relax
%   \providecommand{\doi}[1]{doi: #1}\else
%   \providecommand{\doi}{doi: \begingroup \urlstyle{rm}\Url}\fi
%
% \bibitem[Pakin(2023)]{P2023TFP}
% Scott Pakin.
% \newblock {The \textbf{filecontents} package.}%
% \newblock \emph{CTAN Comprehensive \TeX\ Archive Network}, \mbox{April 2, 2023}.%
% \newblock URL \url{https://ctan.org/pkg/filecontents}%
%
% \end{thebibliography}
% \Finale
\endinput
